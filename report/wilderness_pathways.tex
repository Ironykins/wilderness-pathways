
%% bare_conf.tex
%% V1.4b
%% 2015/08/26
%% by Michael Shell
%% See:
%% http://www.michaelshell.org/
%% for current contact information.
%%
%% This is a skeleton file demonstrating the use of IEEEtran.cls
%% (requires IEEEtran.cls version 1.8b or later) with an IEEE
%% conference paper.
%%
%% Support sites:
%% http://www.michaelshell.org/tex/ieeetran/
%% http://www.ctan.org/pkg/ieeetran
%% and
%% http://www.ieee.org/

%%*************************************************************************
%% Legal Notice:
%% This code is offered as-is without any warranty either expressed or
%% implied; without even the implied warranty of MERCHANTABILITY or
%% FITNESS FOR A PARTICULAR PURPOSE! 
%% User assumes all risk.
%% In no event shall the IEEE or any contributor to this code be liable for
%% any damages or losses, including, but not limited to, incidental,
%% consequential, or any other damages, resulting from the use or misuse
%% of any information contained here.
%%
%% All comments are the opinions of their respective authors and are not
%% necessarily endorsed by the IEEE.
%%
%% This work is distributed under the LaTeX Project Public License (LPPL)
%% ( http://www.latex-project.org/ ) version 1.3, and may be freely used,
%% distributed and modified. A copy of the LPPL, version 1.3, is included
%% in the base LaTeX documentation of all distributions of LaTeX released
%% 2003/12/01 or later.
%% Retain all contribution notices and credits.
%% ** Modified files should be clearly indicated as such, including  **
%% ** renaming them and changing author support contact information. **
%%*************************************************************************


% *** Authors should verify (and, if needed, correct) their LaTeX system  ***
% *** with the testflow diagnostic prior to trusting their LaTeX platform ***
% *** with production work. The IEEE's font choices and paper sizes can   ***
% *** trigger bugs that do not appear when using other class files.       ***                          ***
% The testflow support page is at:
% http://www.michaelshell.org/tex/testflow/


\documentclass[conference]{IEEEtran}
\usepackage{graphicx}
% Some Computer Society conferences also require the compsoc mode option,
% but others use the standard conference format.
%
% If IEEEtran.cls has not been installed into the LaTeX system files,
% manually specify the path to it like:
% \documentclass[conference]{../sty/IEEEtran}





% Some very useful LaTeX packages include:
% (uncomment the ones you want to load)


% *** MISC UTILITY PACKAGES ***
%
%\usepackage{ifpdf}
% Heiko Oberdiek's ifpdf.sty is very useful if you need conditional
% compilation based on whether the output is pdf or dvi.
% usage:
% \ifpdf
%   % pdf code
% \else
%   % dvi code
% \fi
% The latest version of ifpdf.sty can be obtained from:
% http://www.ctan.org/pkg/ifpdf
% Also, note that IEEEtran.cls V1.7 and later provides a builtin
% \ifCLASSINFOpdf conditional that works the same way.
% When switching from latex to pdflatex and vice-versa, the compiler may
% have to be run twice to clear warning/error messages.






% *** CITATION PACKAGES ***
%
%\usepackage{cite}
% cite.sty was written by Donald Arseneau
% V1.6 and later of IEEEtran pre-defines the format of the cite.sty package
% \cite{} output to follow that of the IEEE. Loading the cite package will
% result in citation numbers being automatically sorted and properly
% "compressed/ranged". e.g., [1], [9], [2], [7], [5], [6] without using
% cite.sty will become [1], [2], [5]--[7], [9] using cite.sty. cite.sty's
% \cite will automatically add leading space, if needed. Use cite.sty's
% noadjust option (cite.sty V3.8 and later) if you want to turn this off
% such as if a citation ever needs to be enclosed in parenthesis.
% cite.sty is already installed on most LaTeX systems. Be sure and use
% version 5.0 (2009-03-20) and later if using hyperref.sty.
% The latest version can be obtained at:
% http://www.ctan.org/pkg/cite
% The documentation is contained in the cite.sty file itself.






% *** GRAPHICS RELATED PACKAGES ***
%
\ifCLASSINFOpdf
  % \usepackage[pdftex]{graphicx}
  % declare the path(s) where your graphic files are
  % \graphicspath{{../pdf/}{../jpeg/}}
  % and their extensions so you won't have to specify these with
  % every instance of \includegraphics
  % \DeclareGraphicsExtensions{.pdf,.jpeg,.png}
\else
  % or other class option (dvipsone, dvipdf, if not using dvips). graphicx
  % will default to the driver specified in the system graphics.cfg if no
  % driver is specified.
  % \usepackage[dvips]{graphicx}
  % declare the path(s) where your graphic files are
  % \graphicspath{{../eps/}}
  % and their extensions so you won't have to specify these with
  % every instance of \includegraphics
  % \DeclareGraphicsExtensions{.eps}
\fi
% graphicx was written by David Carlisle and Sebastian Rahtz. It is
% required if you want graphics, photos, etc. graphicx.sty is already
% installed on most LaTeX systems. The latest version and documentation
% can be obtained at: 
% http://www.ctan.org/pkg/graphicx
% Another good source of documentation is "Using Imported Graphics in
% LaTeX2e" by Keith Reckdahl which can be found at:
% http://www.ctan.org/pkg/epslatex
%
% latex, and pdflatex in dvi mode, support graphics in encapsulated
% postscript (.eps) format. pdflatex in pdf mode supports graphics
% in .pdf, .jpeg, .png and .mps (metapost) formats. Users should ensure
% that all non-photo figures use a vector format (.eps, .pdf, .mps) and
% not a bitmapped formats (.jpeg, .png). The IEEE frowns on bitmapped formats
% which can result in "jaggedy"/blurry rendering of lines and letters as
% well as large increases in file sizes.
%
% You can find documentation about the pdfTeX application at:
% http://www.tug.org/applications/pdftex





% *** MATH PACKAGES ***
%
%\usepackage{amsmath}
% A popular package from the American Mathematical Society that provides
% many useful and powerful commands for dealing with mathematics.
%
% Note that the amsmath package sets \interdisplaylinepenalty to 10000
% thus preventing page breaks from occurring within multiline equations. Use:
%\interdisplaylinepenalty=2500
% after loading amsmath to restore such page breaks as IEEEtran.cls normally
% does. amsmath.sty is already installed on most LaTeX systems. The latest
% version and documentation can be obtained at:
% http://www.ctan.org/pkg/amsmath





% *** SPECIALIZED LIST PACKAGES ***
%
%\usepackage{algorithmic}
% algorithmic.sty was written by Peter Williams and Rogerio Brito.
% This package provides an algorithmic environment fo describing algorithms.
% You can use the algorithmic environment in-text or within a figure
% environment to provide for a floating algorithm. Do NOT use the algorithm
% floating environment provided by algorithm.sty (by the same authors) or
% algorithm2e.sty (by Christophe Fiorio) as the IEEE does not use dedicated
% algorithm float types and packages that provide these will not provide
% correct IEEE style captions. The latest version and documentation of
% algorithmic.sty can be obtained at:
% http://www.ctan.org/pkg/algorithms
% Also of interest may be the (relatively newer and more customizable)
% algorithmicx.sty package by Szasz Janos:
% http://www.ctan.org/pkg/algorithmicx




% *** ALIGNMENT PACKAGES ***
%
%\usepackage{array}
% Frank Mittelbach's and David Carlisle's array.sty patches and improves
% the standard LaTeX2e array and tabular environments to provide better
% appearance and additional user controls. As the default LaTeX2e table
% generation code is lacking to the point of almost being broken with
% respect to the quality of the end results, all users are strongly
% advised to use an enhanced (at the very least that provided by array.sty)
% set of table tools. array.sty is already installed on most systems. The
% latest version and documentation can be obtained at:
% http://www.ctan.org/pkg/array


% IEEEtran contains the IEEEeqnarray family of commands that can be used to
% generate multiline equations as well as matrices, tables, etc., of high
% quality.




% *** SUBFIGURE PACKAGES ***
%\ifCLASSOPTIONcompsoc
%  \usepackage[caption=false,font=normalsize,labelfont=sf,textfont=sf]{subfig}
%\else
%  \usepackage[caption=false,font=footnotesize]{subfig}
%\fi
% subfig.sty, written by Steven Douglas Cochran, is the modern replacement
% for subfigure.sty, the latter of which is no longer maintained and is
% incompatible with some LaTeX packages including fixltx2e. However,
% subfig.sty requires and automatically loads Axel Sommerfeldt's caption.sty
% which will override IEEEtran.cls' handling of captions and this will result
% in non-IEEE style figure/table captions. To prevent this problem, be sure
% and invoke subfig.sty's "caption=false" package option (available since
% subfig.sty version 1.3, 2005/06/28) as this is will preserve IEEEtran.cls
% handling of captions.
% Note that the Computer Society format requires a larger sans serif font
% than the serif footnote size font used in traditional IEEE formatting
% and thus the need to invoke different subfig.sty package options depending
% on whether compsoc mode has been enabled.
%
% The latest version and documentation of subfig.sty can be obtained at:
% http://www.ctan.org/pkg/subfig




% *** FLOAT PACKAGES ***
%
%\usepackage{fixltx2e}
% fixltx2e, the successor to the earlier fix2col.sty, was written by
% Frank Mittelbach and David Carlisle. This package corrects a few problems
% in the LaTeX2e kernel, the most notable of which is that in current
% LaTeX2e releases, the ordering of single and double column floats is not
% guaranteed to be preserved. Thus, an unpatched LaTeX2e can allow a
% single column figure to be placed prior to an earlier double column
% figure.
% Be aware that LaTeX2e kernels dated 2015 and later have fixltx2e.sty's
% corrections already built into the system in which case a warning will
% be issued if an attempt is made to load fixltx2e.sty as it is no longer
% needed.
% The latest version and documentation can be found at:
% http://www.ctan.org/pkg/fixltx2e


%\usepackage{stfloats}
% stfloats.sty was written by Sigitas Tolusis. This package gives LaTeX2e
% the ability to do double column floats at the bottom of the page as well
% as the top. (e.g., "\begin{figure*}[!b]" is not normally possible in
% LaTeX2e). It also provides a command:
%\fnbelowfloat
% to enable the placement of footnotes below bottom floats (the standard
% LaTeX2e kernel puts them above bottom floats). This is an invasive package
% which rewrites many portions of the LaTeX2e float routines. It may not work
% with other packages that modify the LaTeX2e float routines. The latest
% version and documentation can be obtained at:
% http://www.ctan.org/pkg/stfloats
% Do not use the stfloats baselinefloat ability as the IEEE does not allow
% \baselineskip to stretch. Authors submitting work to the IEEE should note
% that the IEEE rarely uses double column equations and that authors should try
% to avoid such use. Do not be tempted to use the cuted.sty or midfloat.sty
% packages (also by Sigitas Tolusis) as the IEEE does not format its papers in
% such ways.
% Do not attempt to use stfloats with fixltx2e as they are incompatible.
% Instead, use Morten Hogholm'a dblfloatfix which combines the features
% of both fixltx2e and stfloats:
%
% \usepackage{dblfloatfix}
% The latest version can be found at:
% http://www.ctan.org/pkg/dblfloatfix




% *** PDF, URL AND HYPERLINK PACKAGES ***
%
%\usepackage{url}
% url.sty was written by Donald Arseneau. It provides better support for
% handling and breaking URLs. url.sty is already installed on most LaTeX
% systems. The latest version and documentation can be obtained at:
% http://www.ctan.org/pkg/url
% Basically, \url{my_url_here}.




% *** Do not adjust lengths that control margins, column widths, etc. ***
% *** Do not use packages that alter fonts (such as pslatex).         ***
% There should be no need to do such things with IEEEtran.cls V1.6 and later.
% (Unless specifically asked to do so by the journal or conference you plan
% to submit to, of course. )


% correct bad hyphenation here
\hyphenation{op-tical net-works semi-conduc-tor}


\begin{document}
%
% paper title
% Titles are generally capitalized except for words such as a, an, and, as,
% at, but, by, for, in, nor, of, on, or, the, to and up, which are usually
% not capitalized unless they are the first or last word of the title.
% Linebreaks \\ can be used within to get better formatting as desired.
% Do not put math or special symbols in the title.
\title{Wilderness Exploration and Pathway Formation}


% author names and affiliations
% use a multiple column layout for up to three different
% affiliations
\author{\IEEEauthorblockN{Konrad Aust}
\IEEEauthorblockA{Dept of Computer Science\\
University of Calgary\\
ktaust@ucalgary.ca}
\and
\IEEEauthorblockN{Mark Barley}
\IEEEauthorblockA{Dept of Computer Science\\
University of Calgary\\
mpbarley@ucalgary.ca}}

% conference papers do not typically use \thanks and this command
% is locked out in conference mode. If really needed, such as for
% the acknowledgment of grants, issue a \IEEEoverridecommandlockouts
% after \documentclass

% for over three affiliations, or if they all won't fit within the width
% of the page, use this alternative format:
% 
%\author{\IEEEauthorblockN{Michael Shell\IEEEauthorrefmark{1},
%Homer Simpson\IEEEauthorrefmark{2},
%James Kirk\IEEEauthorrefmark{3}, 
%Montgomery Scott\IEEEauthorrefmark{3} and
%Eldon Tyrell\IEEEauthorrefmark{4}}
%\IEEEauthorblockA{\IEEEauthorrefmark{1}School of Electrical and Computer Engineering\\
%Georgia Institute of Technology,
%Atlanta, Georgia 30332--0250\\ Email: see http://www.michaelshell.org/contact.html}
%\IEEEauthorblockA{\IEEEauthorrefmark{2}Twentieth Century Fox, Springfield, USA\\
%Email: homer@thesimpsons.com}
%\IEEEauthorblockA{\IEEEauthorrefmark{3}Starfleet Academy, San Francisco, California 96678-2391\\
%Telephone: (800) 555--1212, Fax: (888) 555--1212}
%\IEEEauthorblockA{\IEEEauthorrefmark{4}Tyrell Inc., 123 Replicant Street, Los Angeles, California 90210--4321}}


% use for special paper notices
%\IEEEspecialpapernotice{(Invited Paper)}

% make the title area
\maketitle

% As a general rule, do not put math, special symbols or citations
% in the abstract
\begin{abstract}
In this project, we will explore the emergent behaviors of destructive agents moving through randomly generated wilderness terrain. Specifically, we will investigate how pathway networks develop in the landscape, and the extent to which a network of paths eases the difficulties of traversing terrain.
\end{abstract}

% no keywords




% For peer review papers, you can put extra information on the cover
% page as needed:
% \ifCLASSOPTIONpeerreview
% \begin{center} \bfseries EDICS Category: 3-BBND \end{center}
% \fi
%
% For peerreview papers, this IEEEtran command inserts a page break and
% creates the second title. It will be ignored for other modes.
\IEEEpeerreviewmaketitle



\section{Introduction}

\IEEEPARstart{W}{ilderness} terrain is rarely uniformly wild. Most often, it is found carved up with many pre-formed paths and animal trails. How did these paths form? Clearly, they were not engineered, but were the product of some sort of emergence. In this project, we attempt to explore this process by simulating the movement of entities across procedurally generated terrain.
	
	Our simulation models a section of wilderness terrain, through which many different agents move. These agents can represent hikers, all-terrain vehicles, animals, or anything else. Each of these agents has nothing more than a general direction they are traveling, and knowledge of their immediate surroundings. Their goal is to reach their destination through the easiest path they can find. Agents do not have any organization amongst themselves, nor do they communicate directly with one another. Ultimately, the purpose is to see if coherent and complex networks of pathways can emerge from the actions of agents that are not cooperating with one another.
	
	Another topic of investigation is how easy traversing the terrain becomes as the path network reaches maturity. A wild wilderness should 	presumably be a difficult area to move through. Just how much of an effect does an established network of paths have on minimizing this difficulty?
	
	Also, just what is the effect on the terrain itself when these agents come tearing through, trampling everything in their path? Is the destruction they cause enough to destroy ecosystems, or can a wild wilderness be made traversable without compromising its various ecological balances?
	
	We have created a complex system crafted out of relatively simple rules, so there is a lot of potential for completely unforeseen properties to emerge. We hope to find unexpected behaviors in the simulation. As with many experiments, the secondary goal of this simulation is to be happily surprised by it.
	
\section{Simulation Overview}

The simulation was written in NetLogo 5.1. It consists of two major parts: the terrain, and the agents. The terrain is a procedurally generated, grid-based map, and agents are mobile objects that traverse this map.

\subsection{Terrain Generation}
Terrain is generated through a mix of Perlin noise \cite{wiki:perlin} and Voronoi tessellation \cite{wolfram:voronoi}. The Voronoi tessellation divides the map into a number of regions, each corresponding to a different ‘biome’. This biome dictates the difficulty of moving through the tile, as well as the resistance of the tile to being trampled. The perlin noise governs the initial integrity of the tile, which determines how intact the terrain is, and thus how difficult it is to traverse. For example, a snow tile with high integrity can be thought of as deep snow, which is very difficult to navigate, but rapidly becomes very easy to navigate as agents pass through it.

\begin{figure}[h]
\includegraphics[scale=0.25]{image/voronoi}
\includegraphics[scale=0.25]{image/perlin}
\begin{center}
\includegraphics[scale=0.25]{image/both}
\end{center}
\caption{Map Generation Layers. Voronoi Tessellation (top left) and Perlin Noise (top right) are combined to create our final terrain patch. (Bottom)}
\end{figure}

Each type of biome has a difficulty multiplier, and the actual difficulty of a tile is given by

\begin{center}
\verb+difficulty = integrity * diff_mult+
\end{center}

\newpage
Each type of biome also has a deterioration value. When an agent steps on the patch, its integrity is set according to:

\begin{center}
\verb+integrity = old_integrity * det_value+
\end{center}

A darker colored patch in the simulation has a greater integrity value than a lighter colored patch, so when a path is formed, generally it shows up as a line of white patches surrounded by regular terrain.

See table 1 for an overview of the different terrain types we used in our simulation.

\begin{table}[h]
  \begin{center}
    \begin{tabular}{| l | l | l |}
    \hline
    Type & Color & Description \\ \hline \hline
    Forest & Green & Normal movement. Normal Durability \\ \hline
	Rock & Brown & Easy to move through. Very durable. \\ \hline
	Snow & Gray & Difficult to move through. Not durable. \\ \hline
	Underbrush & Blue & Easy to move through. Not durable.\\ \hline
	Jungle & Purple & Difficult to move through. Durable. \\ \hline
	Swamp & Pink & Difficult to move through. Moderately durable. \\ \hline
    \end{tabular}
  \end{center}
\caption{Biome Types}
\end{table}

\subsection{Agent Behavior}
Agents are spawned in at fixed intervals. They appear on one edge of the simulation, and are given a random destination point on another edge of it. They attempt to traverse the terrain to get to their destination, at which point they leave the simulation, and their total weighted distance traveled is plotted.

Agents use A* pathfinding with a limited sight radius to decide which path to take through the terrain. Agents can only determine the difficulty of terrain tiles that are within their sight radius, and assume that all tiles beyond their sight radius have equal difficulty. Each tick, an agent takes a step along its pre-computed path, and at fixed intervals, an agent will re-compute its path.
 
As was mentioned above, as agents move across the terrain, they trample that terrain, reducing its integrity and thus its difficulty to pass through for future agents. The destruction of terrain creates paths that other agents are more likely to follow, not because of any inter-agent communication, but because the paths are now easier to traverse. We can analyze the structure of these paths, along with the weighted distance that each agent travels as the path network is developed.

\subsection{Extension: Terrain Regrowth}
As an extension to the simulation, we implemented a system where, rather than being trampled permanently, terrain can slowly regenerate itself when left alone. Terrain regrows at a user specified rate. This rate scales with the missing integrity of the terrain tile, so regrowth is faster at low integrity, and slower at high integrity. Also, terrain will not regrow for a certain number of ticks after it has been trampled.

We believed that over the long term, this may create a more accurate representation of how paths generate in the wilderness. If agents cease to use a certain path, it can be reclaimed by nature, and so the network of paths can not only grow, it can also shrink.

\section{Results}
Our simulation does in fact produce an emergent network of pathways through the terrain. As you can see, agents tend to create simple networks from edge to edge, as well as forming more complex sections closer to the edges.

\begin{figure}[h]
\includegraphics[scale=0.25]{image/results1}
\includegraphics[scale=0.25]{image/results2}
\caption{On the left, an untouched map. On the right, a map with some established pathway networks.}
\end{figure}

Once an established pathway is found that leads in the same direction as an agent's destination, the agent will walk along it instead of trampling nearby terrain, simply because the trampled section now forms an easier path. By graphing the travel distances of each turtle, and the average trip distance over time, we are able to see a fairly linear decline of difficulty.

\begin{figure}[h]
\includegraphics[scale=0.5]{image/results3}
\caption{Travel Distance and average travel distance as simulation progresses}
\end{figure}

\subsection{Simulation Parameters}

\subsubsection{Turtle Spawn Rates and Population Cap}
The spawning frequency and maximum number of turtles affects the maps formation almost exclusively in the early stages of simulation. Once a network of pathways connects all four edges agents quickly converge to walking across them. Essentially once the pathway has formed these parameters amount to throughput. See figure 4.

\begin{figure}[h]
\begin{center}
\includegraphics[scale=0.4]{image/results4}
\end{center}
\caption{Spawn Rates and Population Cap}
\end{figure}
	
	In the early stages of the simulation this affects the density of our path network, essentially, how many paths through the center of the network are created. This higher density of paths reduce the average cost of traveling over the map as we create more direct pathways. See figure 5.
	
\begin{figure}[h]
\begin{center}
\includegraphics[scale=0.4]{image/results5}
\end{center}
\caption{Alternate path network for the same map.}
\end{figure}

\subsubsection{Agent Sight parameters: Sight Radius - Backtracking}
Backtrack is something that never worked as well as we would have liked. We found that by letting agents turn around they were often stuck in two or three patch loops because as they trample the ground it becomes easier to walk on. I think we could remedy this, but we would need to change our movement system to work based on vectors instead of discrete movements. This would allow our agent’s velocity to respond to the difficulty of their terrain, which we could use to determine how fast an agent can turn around or even just stop. 

Sight radius affects how quickly the agents converge to the already built pathway networks. Having a very low sight radius will generate a more dense pathway network, because they will not be able to find paths that are far apart. With a sight radius of 1, agents just move to the square closer to the destination with lower difficulty. This just tramples the entire map. (See figure 6.)

\begin{figure}[h]
\begin{center}
\includegraphics[scale=0.4]{image/results6}
\end{center}
\caption{An established network of paths from agents with very low sight radius}
\end{figure}

\subsubsection{Terrain Generation Parameters}
These effects were less pronounced on the system's behavior. Noise octaves provided stochastic levels of terrain difficulty, however the higher octaves are affecting the difficulties by such a small margin when compared to other aspects of pathway generation that they have little to no noticeable effect on paths. This could be refined.

Biome count had similar effects. Each biome that must be traversed can be considered a sub-problem of our simulation. Having more biomes means that there are more biomes for a pathway to travel through, generally speaking at least. In a low density network this can have negligible effects because there are only a few pathways anyways, however in higher density path networks this shows structured pathways within each biome. 


\subsection{Extension: Terrain Regrowth}
The terrain regrowth effect either does not affect the system very much, if the regrowth is slow enough to not destroy paths; or it forces agents to make and remake new paths if it's growing faster than the agents can wear it down. This is as expected, we figured it wouldn't have much effect on the regularly used paths, especially considering the rate at which it recovers in the real world; and would more serve to rebuild the nature next to paths. 


\section{Discussion} 
	Our inspiration for this project was 'off the beaten path' trails that are often found to emerge in places where humans and/or animals are presented with some non-descriptive landscape to traverse. Whether it's the lawn in front of math science, or a trail through the woods to the campsite latrine, it's clear that when a rudimentary pathway is formed agents in real life tend to walk the road more traveled. Additionally, every agent in our real world system is acting independent of each other, which makes modeling this system very easy in NetLogo. 
	
	I think our simulations results show several interesting things about the way humans think and work together. Since our simulation can be thought of as graph traversal, it tends to optimize the distance based on difficulty, however this is only a half truth, because as agents walk across the map the damage terrain, which makes the 'optimal' solution change. I think a better way of putting it would be to say that our simulation starts with a naive pathway system created by the first agents that is not very efficient. Over time our simulation serves to optimize the difficulties of the patches it exists on, so our pathway becomes the most efficient version of that specific route possible.
	
Likewise, this is exactly what humans do in real life. For an example consider a group walking through the woods. If the first agent doesn't see the easiest path to their destination, they will pick a less efficient route, deteriorating it slightly in the process. Every agent thereafter will naturally follow this leader, also unaware of the paved highway 20 feet to the left, which is how they would optimize their travel distance. Instead they are optimizing the pass-ability of this secondary pathway, any future agents who walk over it will have an easier time.

This way of doing this actually shows a remarkable level of unconscious forethought and swarm behavior on humanity's behalf, and I find it hilarious how unaware we are of how often these side effects arise from our even most selfish behavior. Everyone who cuts across the Math Science lawn thinks that they are independently trying to optimize their walk to school, this is not considered altruistic, if anything we feel shame for walking on the grass. However this also has the side effect of reducing the difficulty of the short-cut path, and over time we end up with a grass free dirt path that might as well be paved, effectively replacing the old path with a new optimal solution for everyone. Again, the math science lawn is a great example. I do not think I have ever seen a single person leave through the north doors to head east, and walk all the way around the lawn instead of cutting across the dirt path. 

Now I want to be clear here that these extrapolations I am making are not rigorously tested, and should not by any means be taken as fact. My extrapolations about human's intentions stems from my deeply held belief that humans are, generally speaking, incredibly naive; especially when it comes to understanding their own behavior. If you agree with this, our simulation can be thought of as to outline the possibility of manipulating selfish agents into working together; which to me seems easier than getting them to willingly cooperate. 

% An example of a floating figure using the graphicx package.
% Note that \label must occur AFTER (or within) \caption.
% For figures, \caption should occur after the \includegraphics.
% Note that IEEEtran v1.7 and later has special internal code that
% is designed to preserve the operation of \label within \caption
% even when the captionsoff option is in effect. However, because
% of issues like this, it may be the safest practice to put all your
% \label just after \caption rather than within \caption{}.
%
% Reminder: the "draftcls" or "draftclsnofoot", not "draft", class
% option should be used if it is desired that the figures are to be
% displayed while in draft mode.
%
%\begin{figure}[!t]
%\centering
%\includegraphics[width=2.5in]{myfigure}
% where an .eps filename suffix will be assumed under latex, 
% and a .pdf suffix will be assumed for pdflatex; or what has been declared
% via \DeclareGraphicsExtensions.
%\caption{Simulation results for the network.}
%\label{fig_sim}
%\end{figure}

% Note that the IEEE typically puts floats only at the top, even when this
% results in a large percentage of a column being occupied by floats.


% An example of a double column floating figure using two subfigures.
% (The subfig.sty package must be loaded for this to work.)
% The subfigure \label commands are set within each subfloat command,
% and the \label for the overall figure must come after \caption.
% \hfil is used as a separator to get equal spacing.
% Watch out that the combined width of all the subfigures on a 
% line do not exceed the text width or a line break will occur.
%
%\begin{figure*}[!t]
%\centering
%\subfloat[Case I]{\includegraphics[width=2.5in]{box}%
%\label{fig_first_case}}
%\hfil
%\subfloat[Case II]{\includegraphics[width=2.5in]{box}%
%\label{fig_second_case}}
%\caption{Simulation results for the network.}
%\label{fig_sim}
%\end{figure*}
%
% Note that often IEEE papers with subfigures do not employ subfigure
% captions (using the optional argument to \subfloat[]), but instead will
% reference/describe all of them (a), (b), etc., within the main caption.
% Be aware that for subfig.sty to generate the (a), (b), etc., subfigure
% labels, the optional argument to \subfloat must be present. If a
% subcaption is not desired, just leave its contents blank,
% e.g., \subfloat[].


% An example of a floating table. Note that, for IEEE style tables, the
% \caption command should come BEFORE the table and, given that table
% captions serve much like titles, are usually capitalized except for words
% such as a, an, and, as, at, but, by, for, in, nor, of, on, or, the, to
% and up, which are usually not capitalized unless they are the first or
% last word of the caption. Table text will default to \footnotesize as
% the IEEE normally uses this smaller font for tables.
% The \label must come after \caption as always.
%
%\begin{table}[!t]
%% increase table row spacing, adjust to taste
%\renewcommand{\arraystretch}{1.3}
% if using array.sty, it might be a good idea to tweak the value of
% \extrarowheight as needed to properly center the text within the cells
%\caption{An Example of a Table}
%\label{table_example}
%\centering
%% Some packages, such as MDW tools, offer better commands for making tables
%% than the plain LaTeX2e tabular which is used here.
%\begin{tabular}{|c||c|}
%\hline
%One & Two\\
%\hline
%Three & Four\\
%\hline
%\end{tabular}
%\end{table}


% Note that the IEEE does not put floats in the very first column
% - or typically anywhere on the first page for that matter. Also,
% in-text middle ("here") positioning is typically not used, but it
% is allowed and encouraged for Computer Society conferences (but
% not Computer Society journals). Most IEEE journals/conferences use
% top floats exclusively. 
% Note that, LaTeX2e, unlike IEEE journals/conferences, places
% footnotes above bottom floats. This can be corrected via the
% \fnbelowfloat command of the stfloats package.




\section{Conclusion}
The conclusion goes here.




% conference papers do not normally have an appendix


% use section* for acknowledgment
\section*{Acknowledgment}

The authors would like to thank Hugo Elias, for creating a friendly and easy to use guide to the implementation of Perlin Noise \cite{elias:perlin}, Meghendra Singh, for implementing A* pathfinding in netlogo \cite{singh:astar}, and Uri Wilensky for creating Netlogo, and writing an excellent implementation of Voronoi Tessellation for it \cite{wilensky:voronoi} \cite{wilensky:netlogo}.





% trigger a \newpage just before the given reference
% number - used to balance the columns on the last page
% adjust value as needed - may need to be readjusted if
% the document is modified later
%\IEEEtriggeratref{8}
% The "triggered" command can be changed if desired:
%\IEEEtriggercmd{\enlargethispage{-5in}}

% references section

% can use a bibliography generated by BibTeX as a .bbl file
% BibTeX documentation can be easily obtained at:
% http://mirror.ctan.org/biblio/bibtex/contrib/doc/
% The IEEEtran BibTeX style support page is at:
% http://www.michaelshell.org/tex/ieeetran/bibtex/
%\bibliographystyle{IEEEtran}
% argument is your BibTeX string definitions and bibliography database(s)
%\bibliography{IEEEabrv,../bib/paper}
%
% <OR> manually copy in the resultant .bbl file
% set second argument of \begin to the number of references
% (used to reserve space for the reference number labels box)
\begin{thebibliography}{1}
\bibitem{wolfram:voronoi}
Weisstein, Eric W. "Voronoi Diagram." From MathWorld--A Wolfram Web Resource. http://mathworld.wolfram.com/VoronoiDiagram.html

\bibitem{wiki:perlin}
Perlin noise. (2016, March 22). In Wikipedia, The Free Encyclopedia. Retrieved  April 28, 2016, from \verb+https://en.wikipedia.org/w/index.php?title=Perlin_noise+

\bibitem{elias:perlin}
Hugo Elias, "Perlin Noise." (1998, September 14 ). Retrieved April 28, 2016, from \verb+http://freespace.virgin.net/hugo.elias/updates.htm+

\bibitem{singh:astar}
Meghendra Singh (2014). Astardemo1. Retrieved April 27, 2016, from http://ccl.northwestern.edu/netlogo/models/community/Astardemo1.

\bibitem{wilensky:voronoi}
Wilensky, U. (2006). NetLogo Voronoi model. http://ccl.northwestern.edu/netlogo/models/Voronoi. Center for Connected Learning and Computer-Based Modeling, Northwestern University, Evanston, IL.

\bibitem{wilensky:netlogo}
Wilensky, U. (1999). NetLogo. http://ccl.northwestern.edu/netlogo/. Center for Connected Learning and Computer-Based Modeling, Northwestern University, Evanston, IL.

\end{thebibliography}

% that's all folks
\end{document}


